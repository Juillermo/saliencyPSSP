
%https://jarisaramaki.fi/2018/08/07/ten-tips-for-revising-your-first-draft-paper-writing-for-phd-students-pt-15/amp/?\_\_twitter\_impression=true

%Check that your paper is focused. Choose the point of the paper and its key conclusion before you begin writing, stick to your choices, and write the paper so that the reader gets the point already in the abstract and in the introduction. Leave out results that are not required for supporting the key conclusion, or safely tuck them away in the supplementary information document. When editing, if you feel that your paper loses its focus at some point, take a step back and do a major rewrite.

%Check that the figures tell your story. If you just glance through the figures and skim their captions, do you get the point of the paper and its take-home message? If not, go back and revise—after all, skimming is what most of your readers do.

%Make sure that you take the reader’s hand and lead her through the text with signposts. Or, in other words, check that your writing is not confusing. Writing is, in part, psychology, and it aims to modify your reader’s state of mind and to influence what your reader thinks. Feel empathy for your readers and try to get inside their heads, assuming that they know nothing or very little. Your empathy should be reflected at the level of sentences and paragraphs: present familiar things first before moving to new concepts, use leading sentences, glue your sentences together with expressions that guide the reader. Use subheadings. Gently lead the reader from result to result, from paragraph to paragraph, and from sentence to sentence. Never leave it to the reader to connect the dots— always connect them for her. Err on the side of caution: papers where things have been over-explained are rare (if they exist at all), but papers that are all too difficult to follow are frustratingly common.

%Check that you are consistent with nomenclature and notation. Because you have been immersed all too long in the world of your paper, this problem may be hard to spot for you—using an outside reader as a guinea pig is recommended. Problems with notation are easier to detect; problems with naming things are more difficult. Often, while doing research and while conceptualising the paper, there is a number of concepts floating around, and the very same things can have many names in your thinking. Writers of fiction are allowed to use synonyms for variation, but science should be precise: in the final version of your paper, everything should be called by one name only. While it may be evident to you that the thing you call the weight matrix is the same as the thing that was called the correlation matrix in the previous paragraph, your reader quickly gets confused. Never refer to the same thing with multiple terms.

%If you feel that it is impossible to get some part of your text just right, this is often a sign, a message from you to you. When you are stuck with a paragraph that just won’t yield, stop trying to force it. Instead, ask yourself: why is this so difficult? Search your feelings. What would make the paragraph easy to write, what are you missing? Often, you will notice that you are not faced with a writing problem at all—rather, you miss some important piece of understanding. Perhaps your result is not clear after all, or you have not thought enough about some tricky issue and that is why you cannot express it in words. So take a time out, and look for understanding first; the words will come more easily when you have found it.

% ------------------------------------

% Writing should start with Results, Discussion, Methods, Theory, Introductin-Conclusion
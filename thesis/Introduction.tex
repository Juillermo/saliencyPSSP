%% ----------------------------------------------------------------
%% Introduction.tex
%% 0, Aug 10th
%% ---------------------------------------------------------------- 
\chapter{Introduction} \label{Chapter:Introduction}

%The body of the dissertation must not exceed 15,000 words. The entire dissertation (including the front pages, appendices, and bibliography) should not normally exceed 50 pages. 

%Introduction and Conclusion shouldn't include anything that it's not in the other parts.

%Only talk about the project itself at the introduction and NEVER AGAIN.

%Check that your paper is focused. Choose the point of the paper and its key conclusion before you begin writing, stick to your choices, and write the paper so that the reader gets the point already in the abstract and in the introduction. Leave  out results that are not required for supporting the key conclusion, or safely tuck them away in the supplementary information document. When editing, if you feel that your paper loses its focus at some point, take a step back and do a major rewrite.

%Check that there is a clear question and a clear answer—it is all too common to focus on your results and what you have done, instead of stating and then solving a problem. Your results are meaningful only if they solve a meaningful problem. Remember that your paper is neither an account of your work nor a lab diary; it should be a story of an important problem and its solution. Emphasise the problem, both in the Introduction where it should really stand out, and in the Results section and the Discussion. Make it clear to the reader how each result contributes to solving the problem, and what the implications of solving the problem are.

%\begin{enumerate}
%\item What am I going to prove?
%\item Why does this matter?
%\end{enumerate}
%% --------------------------------------------------------



%% --------------------------------------------------------
\tref{Table:tabex} illustrates the results of my work.
\begin{table}[!htb]
  \centering
  \begin{tabular}{cc}
  \toprule
  \textbf{Training Error} & \textbf{Testing Error}\\
  \midrule
  0 & $\infty$\\
  \bottomrule
  \end{tabular}
  \caption{The Results}
  \label{Table:tabex}
\end{table}

\fref{Figure:figex} shows why this is the case.
\begin{figure}[!htb]
  \centering
  \includegraphics[width=8cm]{figure}
  \caption{A colourful picture (reproduced from / adapted from).}
  \label{Figure:figex}
\end{figure}

This page shows you a subfigure example in \fref{Figure:figsubex}.
\begin{figure}[!htb]
  \centering
  \subfigure[The left caption]{
    \includegraphics[width=4.2cm]{figure}
    \label{Figure:figsubex:left}
  }
  \subfigure[The right caption]{
    \includegraphics[width=4.2cm]{figure}
    \label{Figure:figsubex:right}
  }
  \caption{A doubly colourful picture.}
  \label{Figure:figsubex}
\end{figure}

%% ----------------------------------------------------------------
%% Thesis.tex
%% ---------------------------------------------------------------- 
\documentclass{ecsthesis}      % Use the Thesis Style
\graphicspath{{Figures/}}   % Location of your graphics files
\usepackage{natbib}            % Use Natbib style for the refs.
\hypersetup{colorlinks=true}   % Set to false for black/white printing
\input{Definitions}            % Include your abbreviations
%% ----------------------------------------------------------------
\begin{document}
\frontmatter
\title      {Understanding Convolutional Neural Networks on Protein Secondary Structure Prediction / Saliency Map Analysis on Protein Secondary Structure Prediction}
\authors    {\texorpdfstring
             {\href{mailto:grm1g17@soton.ac.uk}{Guillermo Romero Moreno}}
             {Guillermo Romero Moreno}
            }
\addresses  {\groupname\\\deptname\\\univname}
\date       {\today}
\subject    {}
\keywords   {}
%\maketitle
%% ----------------------------------------------------------------
%\begin{abstract}
%Check that the abstract follows the hourglass structure: broad context, narrower context, the research question, your result, implications of your result on your (sub)field, its broader implications. Also, do make sure that your abstract is as jargon-free as possible: only use words that most readers can understand.

%Check that your paper is focused. Choose the point of the paper and its key conclusion before you begin writing, stick to your choices, and write the paper so that the reader gets the point already in the abstract and in the introduction. Leave  out results that are not required for supporting the key conclusion, or safely tuck them away in the supplementary information document. When editing, if you feel that your paper loses its focus at some point, take a step back and do a major rewrite.
%\end{abstract}
\tableofcontents
%\listoffigures
%\listoftables
%\lstlistoflistings
%\listofsymbols{ll}{$w$ & The weight vector}
%\acknowledgements{Thanks to no one.}
%\dedicatory{To \dots}
%% ----------------------------------------------------------------
\mainmatter

%https://jarisaramaki.fi/2018/08/07/ten-tips-for-revising-your-first-draft-paper-writing-for-phd-students-pt-15/amp/?\_\_twitter\_impression=true

%Check that your paper is focused. Choose the point of the paper and its key conclusion before you begin writing, stick to your choices, and write the paper so that the reader gets the point already in the abstract and in the introduction. Leave out results that are not required for supporting the key conclusion, or safely tuck them away in the supplementary information document. When editing, if you feel that your paper loses its focus at some point, take a step back and do a major rewrite.

%Check that the figures tell your story. If you just glance through the figures and skim their captions, do you get the point of the paper and its take-home message? If not, go back and revise—after all, skimming is what most of your readers do.

%Make sure that you take the reader’s hand and lead her through the text with signposts. Or, in other words, check that your writing is not confusing. Writing is, in part, psychology, and it aims to modify your reader’s state of mind and to influence what your reader thinks. Feel empathy for your readers and try to get inside their heads, assuming that they know nothing or very little. Your empathy should be reflected at the level of sentences and paragraphs: present familiar things first before moving to new concepts, use leading sentences, glue your sentences together with expressions that guide the reader. Use subheadings. Gently lead the reader from result to result, from paragraph to paragraph, and from sentence to sentence. Never leave it to the reader to connect the dots— always connect them for her. Err on the side of caution: papers where things have been over-explained are rare (if they exist at all), but papers that are all too difficult to follow are frustratingly common.

%Check that you are consistent with nomenclature and notation. Because you have been immersed all too long in the world of your paper, this problem may be hard to spot for you—using an outside reader as a guinea pig is recommended. Problems with notation are easier to detect; problems with naming things are more difficult. Often, while doing research and while conceptualising the paper, there is a number of concepts floating around, and the very same things can have many names in your thinking. Writers of fiction are allowed to use synonyms for variation, but science should be precise: in the final version of your paper, everything should be called by one name only. While it may be evident to you that the thing you call the weight matrix is the same as the thing that was called the correlation matrix in the previous paragraph, your reader quickly gets confused. Never refer to the same thing with multiple terms.

%If you feel that it is impossible to get some part of your text just right, this is often a sign, a message from you to you. When you are stuck with a paragraph that just won’t yield, stop trying to force it. Instead, ask yourself: why is this so difficult? Search your feelings. What would make the paragraph easy to write, what are you missing? Often, you will notice that you are not faced with a writing problem at all—rather, you miss some important piece of understanding. Perhaps your result is not clear after all, or you have not thought enough about some tricky issue and that is why you cannot express it in words. So take a time out, and look for understanding first; the words will come more easily when you have found it.

% ------------------------------------

% Writing should start with Results, Discussion, Methods, Theory, Introductin-Conclusion
%% ----------------------------------------------------------------
%% Introduction.tex
%% 0, Aug 10th
%% ---------------------------------------------------------------- 
\chapter{Introduction} \label{Chapter:Introduction}

%The body of the dissertation must not exceed 15,000 words. The entire dissertation (including the front pages, appendices, and bibliography) should not normally exceed 50 pages. 

%Introduction and Conclusion shouldn't include anything that it's not in the other parts.

%Only talk about the project itself at the introduction and NEVER AGAIN.

%Check that your paper is focused. Choose the point of the paper and its key conclusion before you begin writing, stick to your choices, and write the paper so that the reader gets the point already in the abstract and in the introduction. Leave  out results that are not required for supporting the key conclusion, or safely tuck them away in the supplementary information document. When editing, if you feel that your paper loses its focus at some point, take a step back and do a major rewrite.

%Check that there is a clear question and a clear answer—it is all too common to focus on your results and what you have done, instead of stating and then solving a problem. Your results are meaningful only if they solve a meaningful problem. Remember that your paper is neither an account of your work nor a lab diary; it should be a story of an important problem and its solution. Emphasise the problem, both in the Introduction where it should really stand out, and in the Results section and the Discussion. Make it clear to the reader how each result contributes to solving the problem, and what the implications of solving the problem are.

%\begin{enumerate}
%\item What am I going to prove?
%\item Why does this matter?
%\end{enumerate}
%% --------------------------------------------------------



%% --------------------------------------------------------
\tref{Table:tabex} illustrates the results of my work.
\begin{table}[!htb]
  \centering
  \begin{tabular}{cc}
  \toprule
  \textbf{Training Error} & \textbf{Testing Error}\\
  \midrule
  0 & $\infty$\\
  \bottomrule
  \end{tabular}
  \caption{The Results}
  \label{Table:tabex}
\end{table}

\fref{Figure:figex} shows why this is the case.
\begin{figure}[!htb]
  \centering
  \includegraphics[width=8cm]{figure}
  \caption{A colourful picture (reproduced from / adapted from).}
  \label{Figure:figex}
\end{figure}

This page shows you a subfigure example in \fref{Figure:figsubex}.
\begin{figure}[!htb]
  \centering
  \subfigure[The left caption]{
    \includegraphics[width=4.2cm]{figure}
    \label{Figure:figsubex:left}
  }
  \subfigure[The right caption]{
    \includegraphics[width=4.2cm]{figure}
    \label{Figure:figsubex:right}
  }
  \caption{A doubly colourful picture.}
  \label{Figure:figsubex}
\end{figure}

%% ----------------------------------------------------------------
%% Theory.tex
%% ---------------------------------------------------------------- 
\chapter{Theory} \label{Chapter: Theory}

Check that you provide enough background information: your reader does not know what you know. Assuming that your reader knows much more than you and therefore omitting background information is a very common problem with students.

Many students seem to think that they know little while everyone else knows a lot—therefore they shouldn’t explain things that everyone probably already knows. It is only later in their careers when they realise that no-one really knows that much! Besides, there will be readers from adjacent (sub)fields and readers who are just learning the tricks of the trade. Use a colleague who works on something slightly different than you as a test reader—ask her which parts of the text are hard to follow, and revise accordingly.

\begin{enumerate}
\item Setting the scene
\item Underlying information
\end{enumerate}
%% ----------------------------------------------------------------
%% Methods.tex
%% 1113
%% ---------------------------------------------------------------- 
\chapter{Methods} \label{Chapter: Methods}

%Parameters are defined in Results (not in Methods)
%It does not matter if you don't include all the methods, only the ones for the final results.

%Check that you provide enough background information: your reader does not know what you know. Assuming that your reader knows much more than you and therefore omitting background information is a widespread problem with students. A typical example would be a Methods section that directly launches into what you have done without first telling why. Although it is evident to you that to get from A to B you need to do X, this is probably far less obvious to the reader. If you only tell the reader that you did X, she is confused. Why did you do X? Never assume that the reader knows your motivation, or the details of every method you used, or why your research question is relevant. Tell her.

%Plagiarism: use of code (this includes open source code!) that was not written by you, without acknowledgement of the source

%Check that the reader can replicate your results. Verify that your Methods section (and the supplementary sections if any) contains everything that the reader needs to know. Also, check that you provide links to your code and your data if it can be released without violating anyone’s privacy.

%\begin{enumerate}
%\item How am I going to measure it?
%\item How am I going to prove it?
%\end{enumerate}

Most of the computation work is performed on the \textit{IRIDIS High-Performance Computing Facility}, with the help of associated support services at the university. They make use of NVIDIA\textregistered$ $ GeForce GTX 1080 Ti Graphical Processing Units (\textit{GPU}s). The use of the GPUs has been possible thanks to the CUDA 9.0 toolkit\footnote{\url{https://developer.nvidia.com/cuda-toolkit}}.

\section{Dataset}
For this project, the database produced and made public by \cite{Zhou2014} is used\footnote{It can be accessed at \url{https://www.princeton.edu/\~7Ejzthree/datasets/ICML2014/}}. This database has been taken as the flagship benchmark since its release and making use of it allows fair comparisons with most state-of-the-art algorithms\footnote{See section \ref{sect:HoF}}. This dataset includes two sub-sets (training and test) of proteins that come from different sources in order to ensure that the test set is composed of completely new samples. For this same reason, they filtered the training set, stripping out every sequence that held 25\% or more similarity with any protein from the test set.
The training set was obtained from PISCES server (\cite{Wang2003})\footnote{\url{http://dunbrack.fccc.edu/Guoli/PISCES\_OptionPage.php}} from a date before January 2014, which was in time culled from the Protein Data Bank (PDB) (\cite{Berman2003}).
It has an original size of 6133 proteins and 5534 after the filtering. The test set comes from the CB513 dataset \cite{Cuff1999} and includes 514 sequences\footnote{Originally 513, but the last one was split in two since it was longer than the 700 amino-acids limit.}.

Protein sequences can be up to 700 amino acids long (with an average of about 214) and have already been pre-processed by \cite{Zhou2014}, with one-hot encoded inputs for the 21 one types of amino-acid along with a 21-long vector of the \textit{pssm} values and their one-hot encoded secondary structure Q8 classes. The appearance frequencies of the eight classes in both datasets are shown in Figure \ref{fig:targetfreq}.

A subset of 256 proteins was taken out of the training set and used for validation, leaving 5278 proteins for the training. This splitting is common in previous papers (\cite{Zhou2014,Sønderby2014,Busia2017,Jurtz2017,Hattori2017}).

\begin{table}[h]
	\centering
	\includegraphics[width=0.5\linewidth]{targetfreq}
	\caption{Target frequencies on the CB6133 (left) and the CB513 (right). Table reproduced from \cite{Hattori2017}.}
	\label{fig:targetfreq}
\end{table}


\section{Network of study}\label{sect:network}

The interpretability methods could be applied to any state-of-the-art network since these would provide the most refined information about the secondary structure problem. However, in order to reach peak accuracies, these models include incredibly huge models that are not handy to work on (especially when it comes to computational time). For this reason, a simpler network is produced and employed for this analysis, with the hope that it will not lose generality over more complex networks.

The network that is used in this study has been built and trained using the open-source code developed by \cite{Jurtz2017}\footnote{Accesible at \url{https://github.com/vanessajurtz/lasagne4bio}.} on the \textit{Lasagne} framework (\cite{Dieleman2015}). While most training configurations are preserved (cross-entropy with L2 regularisation as the error function, batch normalisation, uniform glorot initialisation, mini-batches of size 64, RMSprop rule updates, gradient clipping), the network architecture itself is completely rebuilt. The weights for the final model are the ones from the epoch that has the best performance on the validation set.

The network is composed of three successive convolutional networks and a dense layer on top. Each of the convolutional layers contains three sets of filters of size 3, 5 and 7, respectively, with 16 filters per size. There are skip connections that by-pass every convolutional network in the same fashion as \textit{ResNet}(\cite{He2015}), so the dense layer gets the concatenation of the raw input along with the outputs from the first, second and third layer altogether.
The dense layer has 200 neurons and is connected to a \textit{soft-max} layer that gives the output (secondary structure prediction). The convolution operation is carried out with padding at each end of the sequence to preserve the length of the sequences throughout the convolution operations and produce in the final step one label per position.

% CHeck number of weights at the dense layer

The total window size for this network is 19, meaning that for making a single secondary structure classification the network obtains information from 9 adjacent positions at each side. Although some authors claim that the network should capture long interactions between amino-acids (\cite{Li2016,Lin2016,Hattori2017,Heffernan2017}), it has been proven that most relevant information comes from the local environment (\cite{Busia2017}), so the limited window size should not impinge the model. This fact also supports the idea that the skip connections are beneficial since they avoid the most local information to be ``washed out" in upper layers (\cite{Busia2017}).

% TRy network with a total window size of 43, as Busia did
% TRy with data augmentation, training with the sequences flipped


\section{Feature visualization}
First order filters. They only show the first layer of feature extraction.
Optimization maximisation. Bound to give unreal data. Not credible priors.


\section{Saliency maps} \label{sect:saliency}

% SAliency maps pre or post softmax?

Saliency maps have been calculated by the conventional technique of computing the gradient of the output with respect to the inputs and multiplying it by the value of the input itself (\cite{Shrikumar2016}). A significant difference between this work and most previous papers that make use of saliency maps is that they perform many-to-one classification (one output class per sequence/image), here the classification task is many-to-many (each position of the sequence is assigned a class), producing many saliency maps for a single sequence. More specifically, every single position produces a saliency map that contains the influence of the 42-size input (21 amino-acids plus 21 \textit{pssm} scores) onto the 8-size soft-max output. This information covers the 19 positions surrounding the one being predicted, ending up with a saliency map with total dimension 8x42x19. They are computed using Theano framework (\cite{TheTheanoDevelopmentTeam2016}) since it allows automatic differentiation of symbolical expressions and the use of GPUs.

\subsection{Extracting information from saliency maps}
The presence of overlapping saliency maps allows for different ways of aggregating them and extracting meaningful information. In order to obtain information for a specific sequence, the overlapping saliency maps of such sequence could just be added up, resulting in a single, long \textit{sequence-specific} saliency map of size 8x42x$l$.

By changing the focus to general information of the network behaviour, the sheer addition of all saliency maps produces a single 8x42x19 map with an average behaviour. From this map we could extract information about the general behaviour of a particular class (\textit{class-specific} saliency map) or a certain amino-acid (\textit{pssm-specific} saliency map). If a more fine-grained inspection is desired, clustering techniques can be used, and every cluster creates their independent 8x42x19 representative map by addition of all their components.

%Problems with saliency aggregation %INclude my plots with the sliding saliencies
%On the right, saliencies of subsequent single positions. Each line of a plot represents one amino acid. Each plot corresponds to one position. X axis is the window size (19)

%Motifs slide through the saliencies, which makes:
%Sheer aggregation to capture the sliding effect
%Clustering of full saliencies useless 

%Solution:
%By aggregating each saliency on its window length (right), the motifs are preserved %INclude my plots with the sliding window aggregations
%Clustering on the saliencies (cosine) in the positions with a high prediction for one class can show different kinds of motifs that activate that class

%\subsection{Clustering on saliencies}
%Using the per-class window-aggregated version of individual saliency maps (4 dimensions left)
%Cosine distance metric.
%Show either all profiles per-cluster (3 dimensions) or aggregated profiles (2 dimensions)

\section{Open-source}

As it is the standard practice in both the bioinformatics and machine learning research communities, all the code from this project has been released as open-source in the web platform GitHub and can be accessed through the URL \url{https://github.com/Juillermo/lasagne4bio}, with the hope that the tools here developed can be of use for future research.

%% ----------------------------------------------------------------
%% Results.tex
%% ---------------------------------------------------------------- 
\chapter{Results \& Discussion} \label{Chapter: Results}

% Parameters are defined in Results (not in Methods)

% Report is a presentation of the final results, not the whole process.

% Draw inferences from proofs

% Check that your paper is focused. Choose the point of the paper and its key conclusion before you begin writing, stick to your choices, and write the paper so that the reader gets the point already in the abstract and in the introduction. Leave  out results that are not required for supporting the key conclusion, or safely tuck them away in the supplementary information document. When editing, if you feel that your paper loses its focus at some point, take a step back and do a major rewrite.

% Check that there is a clear question and a clear answer—it is all too common to focus on your results and what you have done, instead of stating and then solving a problem. Your results are meaningful only if they solve a meaningful problem. Remember that your paper is neither an account of your work nor a lab diary; it should be a story of an important problem and its solution. Emphasise the problem, both in the Introduction where it should really stand out, and in the Results section and the Discussion. Make it clear to the reader how each result contributes to solving the problem, and what the implications of solving the problem are.

%\begin{enumerate}
%\item What did I find?
%\item Let me describe the best bits
%\item Here is my analysis and proof of it
%\item Here's my conclusion(s) about it
%\end{enumerate}

\section{Network performance analysis}
% SHow accuracy reached
% SHow sequence accuracy distributions, both sorted per-accuracy and sorted per-length
% CAp on 300 sequence length
% DIscuss about the systematically lower ones. Suggest using SwissProt
Refer to \ref{fig:targetfreq}.

\section{Feature visualization}

	\subsection{First layer filters}
	
	\subsection{Saliency maps on layers?}


\section{Saliency maps on inputs}
Talk about dimensions: saliency value, class, sequences, positions, window, amino-acids, aa/pssm (7 dimensions). From now on, individual saliency maps will refer to the saliency map of one sequence-position

Options (all options could be applied either to aa or to pssm, 6 dimensions left)
% PRove that saliencies on pssm are much more powerful than saliencies on aas
	
	\subsection{Saliency maps on single sample sequences}
	% SHow per-class samples of this instances (either graph form, SeqLogo form) (either per-position or heat-map)
	analyse per-class single sequence (sample-based) (4 dimensions left): aa-aggregated (3 dimensions) and added in the sequence position, either respecting each position (3 dimensions) or aggregating them in a heat map (2 dimensions)

	\subsection{Sheer addition}
	aggregate all individual saliency maps % SHould I try to compare my results with known motifs? Absolutely yes!
	sheer addition (4 dimensions left): class-aggregated (3 dimensions), aa-aggregated (3 dimensions), or class+aa-aggregated (2 dimensions)
		\subsubsection{Per-aminoacid and class aggregations}
		% INclude per-class aggregations (absolute value, aa aggregated) -> class profile (flatness, distribution, etc)
		% XAxis labels should be centered on 0 and going to +-9
		
		\subsubsection{Per-class aggregations}
		% INCLUDE per-class aggregations (either graph form, SeqLogo form, or both). Include just a few and leave the others as supplementary materials.
		First thing to notice: pssm is way more relevant than one-hot encoded aas. No wonder, it learns faster.

		\subsubsection{Per-aminoacid aggregations}
		% INclude per-aminoacid aggregations? (Same, just a few and the others as supplementary)

	\subsection{Clustering techniques}
	aggregate all individual saliency maps % SHould I try to compare my results with known motifs? Absolutely yes!
	clustering (5 dimensions left)
	Using the per-class window-aggregated version of individual saliency maps (4 dimensions left)
	Cosine distance metric.
	Show either all profiles per-cluster (3 dimensions), or aggregated profiles (2 dimensions)

%% ----------------------------------------------------------------
%% Discussion.tex
%% ---------------------------------------------------------------- 
\chapter{Discussion} \label{Chapter: Discussion}

Draw inferences from proofs

Check that there is a clear question and a clear answer—it is all too common to focus on your results and what you have done, instead of stating and then solving a problem. Your results are meaningful only if they solve a meaningful problem. Remember that your paper is neither an account of your work nor a lab diary; it should be a story of an important problem and its solution. Emphasise the problem, both in the Introduction where it should really stand out, and in the Results section and the Discussion. Make it clear to the reader how each result contributes to solving the problem, and what the implications of solving the problem are.

\begin{enumerate}
\item Here is my analysis and proof of it
\item Here's my conclusion(s) about it
\end{enumerate}

\section{Sheer aggregation discussion}

\section{Clustering discussion}
%% ----------------------------------------------------------------
%% Conclusions.tex
%% ---------------------------------------------------------------- 
\chapter{Conclusions} \label{Chapter: Conclusions}

%Introduction and Conclusion shouldn't include anything that it's not in the other parts.
%Objectives should match with conclusions.

%Conclusions mirrors the content of introduction and repeats/summarises the conclusion of the discussions. Nothing new here.

%Check that you end the paper with something worth remembering. This means something concrete. “More research is needed” is a platitude and a vague one at that; better, go for something like “because of the results of this paper, we are now in a position to tackle problem X with method Y, bringing us closer to the ultimate goal of Z”. This is far more concrete and memorable. Endings have power; do not waste this power.

%FUTURE WORK:
% CHeck saliency results when the saliencies for each class are only computed for the positions where that class wins
% TRain a network with a slightly wider window
% CHeck contribution of each layer by computing pre-dense saliency maps
% TRain network without one-hot amino-acids to prove that they are not that relevant
% APplying DeepLift
% DAta augmentation: due to the irrelevance of forward-backwards in the sequences it would be better to have a double database for purely CNNs approaches, with sequences trained in both ways

%\begin{enumerate}
%\item Let me summarise my conclusions about it
%\item And let me state the value of them
%\end{enumerate}

The field of biomedical research requires more than high accuracy; the ability to interpret models is also a valuable asset. Deep learning has increased the first point but fails at the second. Luckily, some interpretability techniques are being developed in the context of image processing and can be applied to other fields. This work has implemented for the first time saliency maps in a structure-to-structure sequential problem, namely, the protein secondary structure prediction problem. While saliency maps in problems with a single output per sequence are simple to interpret, structure-to-structure problems have more dimensions, so further methods for aggregating and inspecting them are also necessary. The aggregating methods shown here focus on sequences, inputs or outputs, uncovering different relations in the data. This type of analysis can be readily understood by experts in the field and provide them with valuable new information.

Preliminary inspection of the saliency map has reassured the superiority of \textit{pssm} scores as inputs, later confirmed by an empirical experiment. Other relevant facts include the skewness of $\alpha$-helices, which are influenced more strongly by the amino-acids coming afterwards. This asymmetry is also present in other classes and it varies depending on the \textit{pssm} we look at. From a class-level perspective, $\alpha$-helices have higher tendency to look at distant amino-acids (inside the limited window of 19); $\beta$-strands have more consideration for further positions as compared to other classes, but they still mainly focus on the close vicinity, so longer interactions does not seem to be captured by the network.

Still, further exploration techniques should be investigated. Other methods of aggregation can be explored since the ones here exposed only show features on a very general level. Clustering of the saliency maps would cover this aspect by including various facets of the features, and all techniques developed here can be applied independently to each of them and thus capture multi-modality. Feature visualisation techniques can also provide with extra valuable information, and their implementation should also be studied.

Finally, other kinds of problems can also be explored, such as backbone angle prediction, where the goal shifts from classification to regression.
\appendix
%% ----------------------------------------------------------------
%% AppendixA.tex
%% ---------------------------------------------------------------- 
\chapter{Saliency aggregations} \label{Chapter:Stuff}
Include all the graphs over here

\backmatter
\bibliographystyle{ecs}
\bibliography{ECS}
\end{document}
%% ----------------------------------------------------------------

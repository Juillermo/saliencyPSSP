%% ----------------------------------------------------------------
%% Conclusions.tex
%% ---------------------------------------------------------------- 
\chapter{Conclusions} \label{Chapter: Conclusions}

Introduction and Conclusion shouldn't include anything that it's not in the other parts.
Objectives should match with conclusions.

Conclusions mirrors the content of introduction and repeats/summarises the conclusion of the discussions. Nothing new here.

Future work.

Check that you end the paper with something worth remembering. This means something concrete. “More research is needed” is a platitude and a vague one at that; better, go for something like “because of the results of this paper, we are now in a position to tackle problem X with method Y, bringing us closer to the ultimate goal of Z”. This is far more concrete and memorable. Endings have power; do not waste this power.

\begin{enumerate}
\item Let me summarise my conclusions about it
\item And let me state the value of them
\end{enumerate}
